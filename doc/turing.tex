\documentclass[a4paper, 10pt]{article}

%% Mandatory stuff
\usepackage[utf8]{inputenc}
\usepackage[T1]{fontenc}
\usepackage[french]{babel}
%% --

\usepackage{amsmath}

%% Bitstream Charter for text and maths
%% with ye Olde french typography
%% Note: fake small caps
\usepackage[uppercase=upright,%
  greeklowercase=upright,%
  bitstream-charter]{mathdesign}
%% --

\usepackage[usenames, dvipsnames]{xcolor}
\usepackage{url}

%% Code listings
\usepackage{listings}
% UTF8 compatibility options
% UTF8 chars are OK in code comment
% but not elsewhere
\lstset{extendedchars=false}
%% Eye candy
\lstdefinestyle{my-lisp}{
  language=Lisp,
  basicstyle=\small\rmfamily,
  tabsize=3,
  breaklines=false,
  showstringspaces=false,
  frame=l,
  columns=fullflexible,
  framerule=2pt,
  rulecolor=\color{black!50}
}
\lstset{style=my-lisp}

\title{Simulateur de machines de Turing}
\author{Florent Marchand de Kerchove}

\begin{document}

\maketitle

%% Intro

Une machine de Turing est une construction volontairement simple,
puisque destinée à être manipulée par un humain, et non par un
automate capable d'effectuer des centaines de transitions par seconde.
Un simulateur est donc assez court à décrire.

Nous avons choisi le Common Lisp pour écrire ce programme, sans raison
pratique particulière ; la justification est plutôt culturelle.  Le
Lisp est un ancêtre parmi les langages de programmation, tout comme
les machines de Turing sont les ancêtres des ordinateurs, et pourtant
tous deux sont loin d'être obsolètes.

\subsection*{Implémentation en Common Lisp}

Pour simuler une machine, il nous faut la représenter en mémoire. En
Lisp, la structure de liste est favorite, mais pas forcément
efficace. Nous utiliserons donc également des tableaux et des vecteurs
fournis par le Common Lisp.

Afin de définir la machine, nous n'avons besoin que de ses transitions
et de son état initial; tous les symboles de l'alphabet d'entrée, de
sortie, et tous les états qui ne sont pas mentionnés dans les
transitions ne seront pas utiles au simulateur, puisque jamais
utilisés par la machine. Si l'on souhaite définir des machines
acceptantes, il nous faudra également préciser la liste des états
finaux. Enfin, nous laissons la possibilité de définir le symbole
blanc, en utilisant le \# par défaut, conventionnellement.

Les règles de transition seront donc passés dans une liste de
vecteurs. Chaque vecteur spécifie une règle de la forme :
\lstinline{#(p a q b x)}, où $p,q$ sont des états, $a,b$ des symboles
et $x \in {"L", "R"}$. Notons que le Lisp n'impose pas de restriction
sur les types. Nos états et symboles peuvent donc être spécifiés par
des entiers, des chaînes de caractères, des listes ou même des
machines de Turing !  Bien sûr, dans la pratique il sera plus aisé
d'utiliser un type atomique comme les entiers ou les chaînes. Par
exemple, la règle suivante : \lstinline{#(0 "a" 1 "b" "R")} remplace
le symbole $a$ par $b$ sur la bande, si l'état courant est 0, et passe
alors à l'état 1, en décalant la tête de lecture sur la droite.

L'état initial est spécifié par le symbole qui le représente dans les
règles de transition, et les états finaux sont passés dans une liste,
qui peut être vide dans le cas d'une machine en mode
calculateur. Enfin le symbole blanc spécial est également librement
choisi, mais nous utiliserons \# dans nos exemples.

Tout ceci nous permet de modéliser la machine, mais il nous manque un
élément crucial : la bande d'entrée et de sortie. Cette bande sera
représentée par un tableau à dimension variable (finie, sauf si vous
disposez d'une machine à mémoire infinie), et la position de la tête
de lecture sera conservée dans un entier qui indice la bande à partir
de zéro.

Ces informations en main, il nous suffit, pour simuler la machine sur
une bande donnée, de faire une boucle qui exécute la transition
suivante selon la donnée présente sous la tête de lecture et l'état
courant, tant qu'une telle transition est possible. Lorsque la boucle
s'arrête, la bande modifiée par le machine est affichée.
Cette boucle contiendra en plus un test qui vérifiera si l'état
courant est un état final, auquel cas le programme affichera le
message «~calcul acceptant~».

Nous ne prenons pas de précautions particulière pour garantir le
«~bon~» fonctionnement de la machine : l'utilisateur est responsable
des machines à simuler, et nous du simulateur. En particulier, c'est à
lui de s'assurer que l'état initial figure bien dans les transitions,
de même pour les éáts finaux. La programme vérifie un point : que la
tête de lecture ne passe pas à gauche de l'origine de la bande, ce qui
n'est pas accepté sur les machines du modèle standard.

\subsection*{Exemples d'utilisation}

Pour tester le simulateur, il faut tout d'abord lancer un interpréteur
de Common Lisp. Nous utilisons celui fourni par
GNU \footnote{\url{http://clisp.cons.org}}. Ensuite, il faut charger
le fichier "turing.lisp" :

\begin{lstlisting}
> (load "turing.lisp")
;; Loading file turing.lisp ...
;; Loaded file turing.lisp
T
\end{lstlisting}

Ensuite, il suffit d'appeler la fonction \textit{run-machine} sur une
machine construite avec la fonction \textit{make-machine} et une bande
retournée \textit{make-tape}, comme ceci :

\begin{lstlisting}
> (run-machine (make-machine '(#(0 "#" 1 0 "R")
                               #(0 0 0 0 "R")) 0)
               (make-tape '(0 0 0)))
"0000"
\end{lstlisting}

Cette machine calcule le successeur du nombre passé en entrée, sous la
forme standard $0^n$. Le premier argument de \textit{make-machine} est
la liste des transitions. La première, \lstinline{(0 "#" 1 0 "R")}, transforme
le \# en 0 si l'état courant est l'état 0, qui passe alors à l'état 1, en
déplaçant la tête de lecture sur la droite. La seconde règle est une
boucle sur l'état 0 qui lit les 0 sans les effacer. Le second argument
de \textit{make-machine} est l'état initial, 0. Les états finaux ne
sont pas utilisés dans cette machine en mode calculateur, donc pas
passés en argument. Le seul argument de \textit{make-tape} est la
donnée d'entrée sur la bande sous forme de liste de symboles de
l'alphabet d'entrée ; ici ce sont 3 zéros. Enfin, l'interpréteur Lisp
renvoie la chaîne "0000" qui représente la bande en fin de calcul
(les \# superflus de gauche et de droite sont omis).

Nous avons décrit dans la suite quelques autre machines élémentaires
pour illustrer le fonctionnement du simulateur.

\lstinputlisting[firstline=6]{machines.lisp}

\subsection*{Code source}

\lstinputlisting{turing.lisp}

\end{document}
